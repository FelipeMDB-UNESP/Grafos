\documentclass[12pt, a4paper]{scrreprt}

% Pacotes básicos
\usepackage[utf8]{inputenc}
\usepackage[portuguese]{babel}
\usepackage{geometry}
\usepackage{hyperref}
\usepackage{graphicx}

\usepackage[style=authoryear]{biblatex}
\addbibresource{referencias.bib} % Nome do arquivo .bib

\usepackage{helvet} % Similar à Arial
\renewcommand{\familydefault}{\sfdefault} % Sans-serif

% Configurações de margens
\geometry{left=2cm, right=2cm, top=2cm, bottom=2cm}

% Configurações de cabeçalhos e rodapés
\usepackage{scrlayer-scrpage}
\pagestyle{scrheadings}

% Definições de cabeçalhos e rodapés
\ihead{Árvores}
\chead{\leftmark}
\ohead{\pagemark}

% Início do documento
\begin{document}
\begin{titlepage}
    \centering
    \begin{figure}[h]
        \centering
        \includegraphics[width=.75\textwidth]{src/logo_unesp.jpg}
        \label{fig:logo_unesp}
    \end{figure}
    \vfill
    \Huge\textbf{Relatório de estudo sobre grafos do \\ tipo árvore}\\[1.5cm]
    \Large\textbf{Grafos e Aplicações}\\[1.5cm]
    \vfill
    \begin{flushleft}
        \textbf{Equipe:}\\
        \hspace{1.5cm}André Luis Dias Nogueira \\ 
        \hspace{1.5cm}Felipe Melchior de Britto \\
        \hspace{1.5cm}Rafael Daiki Kaneko \\
        \hspace{1.5cm}Ryan Hideki Tadeo Guimarães \\
        \hspace{1.5cm}Vitor Marchini Rolisola \\
    \end{flushleft}
    \vfill
    01/08/2024\\
\end{titlepage}

% Sumário
\tableofcontents
\newpage

% Resumo
\chapter{Resumo}

% Introdução
\chapter{Introdução}
Grafos são estruturas fundamentais em teoria dos grafos, utilizadas para modelar uma variedade de problemas em diferentes áreas, desde redes de computadores até genética. Um tipo especial de grafo, conhecido como árvore, apresenta propriedades únicas que tornam essa classe particularmente interessante para estudo. Uma árvore é definida como um grafo não-orientado, conexo e acíclico, o que significa que não possui ciclos e, além disso, qualquer remoção de uma de suas arestas resulta em um grafo desconexo. Essas características permitem que árvores sejam a estrutura mínima necessária para garantir a conectividade entre os vértices de um grafo com o menor número de arestas possíveis, um aspecto crucial para a otimização de recursos em diversos cenários práticos.

A análise de árvores em grafos tem implicações diretas em problemas de interligação, como o fornecimento de redes elétricas, onde o objetivo é minimizar o custo de conexão ao garantir que todas as unidades estejam conectadas. Além disso, árvores também desempenham um papel importante na computação, particularmente em algoritmos de ordenação, como o Heapsort, e na modelagem de genealogias e redes hierárquicas. O presente relatório visa explorar as propriedades matemáticas e aplicativas das árvores, abordando tanto sua definição formal quanto suas extensões, como arborescências e a aplicação em algoritmos de busca.

A introdução a essas ideias será contextualizada com base nas propriedades da conectividade e da aciclicidade, discutindo ainda como árvores podem ser vistas como estruturas mínimas e otimizadas para representação de relações complexas, ao mesmo tempo que mantêm a simplicidade computacional.

% Implementação
\chapter{Implementação}

% Resultados e discussão
\chapter{Resultados e discussão}

% Conclusão
\chapter{Conclusão}

% Referências bibliográficas
\chapter{Referências bibliográficas}
%Aqui está uma citação de um livro \cite{lamport1994latex} e de um artigo \cite{knuth1984tex}.

%\printbibliography % Para imprimir a bibliografia

\end{document}