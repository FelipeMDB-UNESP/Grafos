\documentclass[12pt, a4paper]{scrreprt}

% Pacotes básicos
\usepackage[utf8]{inputenc}
\usepackage[portuguese]{babel}
\usepackage{geometry}
\usepackage{hyperref}
\usepackage{graphicx}
\usepackage{svg}

% Pacotes para citações
\usepackage{csquotes}
\usepackage[backend=biber,style=numeric]{biblatex} 
\addbibresource{referencias.bib} % Nome do arquivo .bib

\usepackage{helvet} % Similar à Arial
\renewcommand{\familydefault}{\sfdefault} % Sans-serif

% Define o espaço entre parágrafos
\setlength{\parskip}{1em} % Ajuste o valor conforme necessário
\setlength{\parindent}{0pt} % Sem recuo

% Configurações de margens
\geometry{left=2cm, right=2cm, top=2cm, bottom=2cm}

% Configurações de cabeçalhos e rodapés
\usepackage{scrlayer-scrpage}
\pagestyle{scrheadings}

% Definições de cabeçalhos e rodapés
\ihead{Árvores}
\chead{\leftmark}
\ohead{\pagemark}

% Início do documento
\begin{document}
\begin{titlepage}
    \centering
    \begin{figure}[h]
        \centering
        \includegraphics[width=.75\textwidth]{src/logo_unesp.jpg}
        \label{fig:logo_unesp}
    \end{figure}
    \vfill
    \Huge\textbf{Relatório de estudo sobre grafos do \\ tipo árvore}\\[1.5cm]
    \Large\textbf{Grafos e Aplicações}\\[1.5cm]
    \vfill
    \begin{flushleft}
        \textbf{Equipe:}\\
        \hspace{1.5cm}André Luis Dias Nogueira \\ 
        \hspace{1.5cm}Felipe Melchior de Britto \\
        \hspace{1.5cm}Rafael Daiki Kaneko \\
        \hspace{1.5cm}Ryan Hideki Tadeo Guimarães \\
        \hspace{1.5cm}Vitor Marchini Rolisola \\
    \end{flushleft}
    \vfill
    01/08/2024\\
\end{titlepage}

% Sumário
\tableofcontents
\newpage

% Resumo
\chapter{Resumo}

% Introdução
\chapter{Introdução}

Grafos são estruturas fundamentais em teoria dos grafos, utilizadas para modelar uma variedade de problemas em diferentes áreas, desde redes de computadores até genética.

Um grafo é uma estrutura matemática usada para modelar relações entre objetos de um conjunto. Ele é composto por dois conjuntos: um conjunto de vértices (ou nós) e um conjunto de arestas (ou arcos) que conectam esses vértices. Os vértices representam os objetos e as arestas representam as relações entre esses objetos. \textsuperscript{\cite{emilio2024grafos}}

\begin{figure}[h]
    \centering
    \includegraphics[width=.75\textwidth]{src/exemplo_simples_grafo.png}
    \label{fig:exemplo de grafo simples}
\end{figure}

Por exemplo, na representação de uma rede social, os vértices podem representar pessoas e as arestas representam as conexões de amizade entre elas.

Um tipo especial de grafo, conhecido como árvore, apresenta propriedades únicas que tornam essa classe particularmente interessante para estudo.
Uma árvore é definida como um grafo não-orientado, conexo e acíclico, o que significa que não possui ciclos e, além disso, qualquer remoção de uma de suas arestas resulta em um grafo desconexo.\textsuperscript{\cite{definicaoarvore}} Então suas características são:

\begin{itemize}
        \item \textbf{Conectividade}: Para qualquer par de vértices \( u \) e \( v \), existe exatamente um caminho que conecta \( u \) e \( v \).
        \item \textbf{Aciclicidade}: O grafo não contém ciclos; ou seja; não é possivel iniciar em im vértice, seguir arestas e retornar ao mesmo vértice sem atravessar arestas repetidamente.
        \item \textbf{Número de arestas}: Se uma árvore possui \( n \) vértices, então ela possui exatamente \( n - 1 \) arestas.
\end{itemize}

\begin{figure}[h]
    \centering
    \includegraphics[width=.75\textwidth]{src/arvore_exemplo.png}
    \label{fig:exemplo de árvore}
\end{figure}

Essas características permitem que árvores sejam a estrutura mínima necessária para garantir a conectividade entre os vértices de um grafo com o menor número de arestas possíveis, um aspecto crucial para a otimização de recursos em diversos cenários práticos.

A análise de árvores em grafos tem implicações diretas em problemas de interligação, como o fornecimento de redes elétricas, onde o objetivo é minimizar o custo de conexão ao garantir que todas as unidades estejam conectadas. Além disso, árvores desempenham um papel importante na computação, particularmente em algoritmos de ordenação, como o Heapsort, e na modelagem de genealogias e redes hierárquicas.


Uma \textbf{árvore binomial} é uma estrutura de dados que representa uma coleção de árvores binomiais. A definição formal de uma árvore binomial é a seguinte\textsuperscript{\cite{definicaoarvorebinomialEllis} \cite{definicaoarvorebinomialTarjan}}:

Uma \textbf{árvore binomial} \( B_k \) é uma árvore que possui as seguintes propriedades:

\begin{itemize}
    \item \textbf{Estrutura Recursiva}: Uma árvore binomial \( B_k \) é composta por \( 2^k \) nós e tem exatamente \( k \) árvores binomiais \( B_{k-1}, B_{k-2}, \ldots, B_0 \) como subárvores. A árvore \( B_k \) é obtida ao unir duas árvores \( B_{k-1} \).
    
    \item \textbf{Propriedades dos Nós}: 
    \begin{itemize}
        \item O nó na raiz de \( B_k \) tem um grau de \( k \) (ou seja, ele possui \( k \) filhos).
        \item A altura de \( B_k \) é \( k \).
        \item A árvore \( B_k \) possui \( 2^k \) folhas.
    \end{itemize}
    
    \item \textbf{Organização dos Nós}: Os nós são organizados de tal forma que os valores dos nós na subárvore esquerda são menores ou iguais ao valor do nó pai, e os valores dos nós na subárvore direita são maiores.
\end{itemize}

\begin{figure}[h]
    \centering
    \includegraphics[width=.75\textwidth]{src/arvore_binomial_ordem_3.png}
    \label{fig:exemplo de árvore binomial}
\end{figure}

As árvores binomiais são particularmente úteis em algoritmos de estrutura de dados, como em filas de prioridade.

m grafo do tipo árvore binomial, que foi implementado para este relatório, é uma estrutura de dados que representa uma coleção de árvores binomiais. A definição formal de uma árvore binomial é a seguinte:

O presente relatório tem o objetivo de explorar as propriedades matemáticas e aplicativas das árvores binomiais, abordando tanto sua definição formal quanto suas extensões, como arborescências e a aplicação em algoritmos de busca.

A introdução a essas ideias será contextualizada com base nas propriedades da conectividade e da aciclicidade, discutindo ainda como árvores binomiais podem ser vistas como estruturas mínimas e otimizadas para representação de relações complexas, ao mesmo tempo que mantêm a simplicidade computacional.

% Implementação
\chapter{Implementação}

% Resultados e discussão
\chapter{Resultados e discussão}

% Conclusão
\chapter{Conclusão}

% Referências bibliográficas
%\chapter{Referências bibliográficas}
%Aqui está uma citação de um livro \cite{lamport1994latex} e de um artigo \cite{knuth1984tex}.

\printbibliography % Para imprimir a bibliografia

\end{document}