\documentclass[12pt, openright, oneside, a4paper, chapter=TITLE, section=TITLE, subsection=TITLE, subsubsection=TITLE, brazil]{abntex2}

% Pacotes básicos
\usepackage{graphicx} % Inclusão de gráficos

\usepackage[table,xcdraw]{xcolor} % tabelasz
\usepackage{booktabs}
% pasta de figuras
\graphicspath{{figuras/}}
% extensões permitidas
\DeclareGraphicsExtensions{.pdf,.eps,.svg,.png,.jpg,.bmp}

\usepackage{microtype} % para melhorias de justificação
\usepackage{amsmath,amssymb,unicode-math} % escrita matematica
\usepackage{verbatim}

% para incluir páginas pdf diretamente no documento
\usepackage{pdfpages}
\usepackage{csquotes}

% Opção 2: notas explicativas no sistema autor-data
\usepackage[backend=biber,
% configuracoes do estilo abnt
style=abnt,
sccite, % sobrenomes em caixa alta
ittitles, % Titulos em italico
citecount, % contar o número de citações
scbib, % biliografia em caixa alta
justify,
noslsn,
repeatfields,
sorting=nty, % ordem alfabetica
]{biblatex}

% ARQUIVO COM AS REFERÊNCIAS BIBLIOGRAFICAS
\addbibresource{referencias.bib}

% ---
% Personalização do estilo biblatex-abnt - por Danny A. V Tonidandel
% Adequa as urls de acordo com normas 6023:2018
\DeclareFieldFormat{url}{\bibstring{urlfrom}\addcolon\addspace \url{#1}}%
\DeclareFieldFormat{urldate}{\bibstring{urlseen}\addcolon\addspace #1}%

% Reseta contadores das notas de rodapé em cada capítulo
\makeatletter
\@addtoreset{footnote}{chapter}
\makeatother

% Pacotes adicionais, usados apenas no âmbito do Modelo Canônico do abnteX2 - podem ser removidos
% ---
% Pacotes adicionais, usados no anexo do modelo de folha de identificação
% ---
% Tabelas
\usepackage{supertabular}
\usepackage{multicol}
\usepackage{multirow}



\usepackage{lipsum}




% Informações de dados para CAPA e FOLHA DE ROSTO
\titulo{Relatório sobre Grafos Hamiltonianos}
\autor{
André Luis Dias Nogueira\\
Felipe Melchior de Britto\\
Rafael Daiki Kaneko\\
Ryan Hideki Tadeo Guimarães\\
Vitor Marchini Rolisola}
\local{Rio Claro}
\data{\the\year}
\orientador{Prof. Emílio Bergamim Júnior}
\instituicao{UNIVERSIDADE ESTADUAL PAULISTA}
\tipotrabalho{Relatório Acadêmico}
\preambulo{Este relatório apresenta a implementação de testes para verificar se grafos aleatórios satisfazem os teoremas hamiltonianos de Dirac, Ore e Bondy-Chvatl, utilizando modelos de grafos com ciclo inicial e arestas adicionadas com probabilidade p. A análise foi realizada para diferentes valores de N e p, com resultados apresentados em gráficos e tabelas.}
% Configurações de aparência do PDF final

% alterando o aspecto da cor azul
%\definecolor{blue}{RGB}{41,5,195}

% informações do PDF
\makeatletter
\hypersetup{
     	%pagebackref=true,
		pdftitle={\@title},
		pdfauthor={\@author},
    	pdfsubject={\imprimirpreambulo},
	    pdfcreator={LaTeX with abnTeX2},
		pdfkeywords={ufop}{latex}{abntex}{decat}{monografia},
		colorlinks=true,       		% false: boxed links; true: colored links
    	linkcolor=blue,          	% color of internal links
    	citecolor=blue,        		% color of links to bibliography
    	filecolor=magenta,      		% color of file links
		urlcolor=blue,
		%bookmarksdepth=4
}
\makeatother

% Espaçamentos entre linhas e parágrafos
% O tamanho do parágrafo é dado por:
\setlength{\parindent}{1.3cm}
% Controle do espaçamento entre um parágrafo e outro:
\setlength{\parskip}{0.2cm}  % tente também \onelineskip

% compila o indice
\makeindex

% Início do documento
\begin{document}

% Retira espaço extra obsoleto entre as frases.
\frenchspacing

% ----------------------------------------------------------
% ELEMENTOS PRÉ-TEXTUAIS
% ----------------------------------------------------------
% \pretextual

% ---
% Capa
% ---
%\imprimircapa
\begin{capa}
    \thispagestyle{empty}
    \centering
    \vspace*{1cm}

    \begin{minipage}{0.2\linewidth}
        \centering
        \includegraphics[width=3cm]{src/unesp.jpg}
    \end{minipage}%
    \hfill
    \begin{minipage}{0.6\linewidth}
        \centering
        {\large \imprimirinstituicao} \\
        {\large ``J\'ulio de Mesquita Filho''} \\
        {\large Instituto de Geociências e Ciências Exatas} \\
        {\large DEMAC} \\
        {\large Ciências da Computação}
    \end{minipage}%
    \hfill
    \begin{minipage}{0.2\linewidth}
        \centering
        \includegraphics[width=2.1cm]{src/igce.jpg}
    \end{minipage}

    \vspace*{1cm}
    {\ABNTEXchapterfont\large\imprimirautor}
    \vspace*{\fill}

    {\ABNTEXchapterfont\bfseries \large \imprimirtitulo}
    \vspace*{\fill}

    {\large\imprimirtipotrabalho}
    \vspace*{\fill}

    {\large\imprimirlocal}, {\large\imprimirdata}
    \vspace*{1cm}
\end{capa}

% Folha de rosto
\imprimirfolhaderosto*

% RESUMO
\setlength{\absparsep}{18pt} % ajusta o espaçamento dos parágrafos do resumo
\begin{resumo}
 \lipsum[1-4]

 \textbf{Palavras-chaves}: latex. abntex. editoração de texto.
\end{resumo}


% SUMÁRIO
%\pdfbookmark[0]{\contentsname}{toc}
\tableofcontents*
%\cleardoublepage
% ---
% Comando para resetar contadores das notas de rodapé
%\makeatletter
%\@addtoreset{footnote}{chapter}
%\makeatother

% ----------------------------------------------------------
% ELEMENTOS TEXTUAIS
% ----------------------------------------------------------

% INTRODUÇÃO
\chapter[Introdução]{Introdução}
A Teoria dos Grafos é uma área essencial da matemática discreta, com aplicações em diversos campos, incluindo ciência da computação, logística, redes de comunicação, biologia computacional e pesquisa operacional. Um conceito particularmente relevante nessa teoria é o de ciclos hamiltonianos, onde se busca um percurso cíclico que passa por todos os vértices de um grafo exatamente uma vez. Grafos que contêm tais ciclos são denominados \textbf{grafos hamiltonianos} e têm implicações práticas em problemas de otimização, como o problema do caixeiro viajante e o roteamento de redes, onde se procura uma rota eficiente que minimize o custo de deslocamento.

\section{Justificativas e Relevância}
O estudo de grafos hamiltonianos ganha importância na medida em que muitos problemas complexos podem ser simplificados pela verificação de hamiltonianidade em suas representações gráficas. Contudo, a determinação exata da presença de ciclos hamiltonianos é um problema computacionalmente difícil (NP-completo). Para contornar essa dificuldade, a teoria propõe critérios suficientes de hamiltonianidade, que, embora não garantam uma solução exata para todos os grafos, oferecem maneiras eficientes de inferir a presença de ciclos hamiltonianos em grafos que satisfaçam certas condições. Os teoremas de \textbf{Dirac}, \textbf{Ore} e \textbf{Bondy-Chvátal} são três desses critérios, cada um propondo condições suficientes que, quando satisfeitas, garantem a hamiltonianidade do grafo. A relevância desses teoremas está no potencial de reduzir significativamente a complexidade do problema da hamiltonianidade, o que tem implicações diretas em áreas como o planejamento urbano e a configuração de redes, onde rotas e conexões precisam ser eficientes e bem estruturadas.

Explorar e comparar os modelos de grafos que satisfaçam esses teoremas em condições variáveis de conexão oferece uma base empírica valiosa para avaliar a aplicabilidade e a robustez de cada critério. Esse estudo também contribui para uma melhor compreensão dos modelos aleatórios de grafos, que frequentemente são usados para simular redes reais, onde a distribuição de conexões segue padrões probabilísticos.

\section{Metodologia}
A metodologia proposta para este estudo envolve a implementação de um conjunto de testes para verificar se um grafo dado satisfaz os critérios de hamiltonianidade estabelecidos pelos teoremas de Dirac, Ore e Bondy-Chvátal. Para isso, serão aplicados algoritmos específicos para cada teorema:
\begin{enumerate}
    \item Teste de Dirac: Será verificado se todos os vértices de um grafo possuem grau $\delta \geq \frac{n}{2}$, sendo $n$ o número de vértices do grafo.
    \item Teste de Ore: Para cada par de vértices não adjacentes $u$ e $v$, será avaliado se a soma dos graus $d(u) + d(v) \geq n$.
    \item Teste de Bondy-Chvátal: Utilizando o método de fechamento do grafo, serão inseridas arestas entre vértices não adjacentes sempre que a soma de seus graus seja pelo menos $n$, e em seguida, será avaliado se o grafo resultante é hamiltoniano.
\end{enumerate}
Os testes serão aplicados a grafos gerados aleatoriamente de acordo com dois modelos:
\begin{itemize}
    \item Modelo Cíclico-Aleatório: Um grafo inicialmente configurado como um ciclo simples de $N$ vértices (o que garante que ele seja hamiltoniano) e, em seguida, arestas adicionais são inseridas entre pares de vértices com uma probabilidade $p$.
    \item Modelo de Erdos-Renyi: Cada par de vértices recebe uma aresta com uma probabilidade fixa $p$, sem uma configuração inicial de ciclo, resultando em grafos com conectividade aleatória.
\end{itemize}
Para cada combinação de $N$ (número de vértices) e $p$ (probabilidade de conexão), serão gerados dez grafos aleatórios. Cada grafo será submetido aos três testes, e os resultados serão organizados em tabelas e gráficos, comparando a frequência com que cada teorema é satisfeito em cada modelo.

\section{Objetivos}
Este estudo possui os seguintes objetivos principais:
\begin{enumerate}
    \item Explorar a Aplicabilidade dos Teoremas de Dirac, Ore e Bondy-Chvátal: Aprofundar a compreensão dos critérios de hamiltonianidade em grafos aleatórios, identificando em que circunstâncias cada teorema é aplicável.
    \item Desenvolver Testes Computacionais para Verificação da Hamiltonianidade: Implementar algoritmos que verifiquem a conformidade de grafos com os três teoremas, de modo a avaliar a eficiência de cada critério como indicador de hamiltonianidade.
    \item Comparar Modelos de Grafos Aleatórios: Examinar a eficácia dos modelos cíclico-aleatório e Erdos-Renyi na produção de grafos que satisfaçam os critérios de hamiltonianidade, e comparar as taxas de grafos hamiltonianos produzidos por cada modelo para diferentes valores de $N$ e $p$.
\end{enumerate}
Este estudo pretende fornecer uma visão prática e teórica sobre os critérios hamiltonianos, contribuindo para o entendimento de sua aplicabilidade e oferecendo uma base empírica para o uso desses critérios na análise e simulação de redes complexas.

\chapter{Desenvolvimento}

\printindex

\end{document}
